\begin{tikzpicture}[node distance=1.5cm and 2.5cm, >=stealth, bend angle=45, auto] % Adjusted default node distances

% Define places with new coordinates for better layout
\placewithtokens{p1}{0,0}{2}        % Resource
\placewithtokens{p2}{5,0}{0}        % Process
\placewithtokens{p3}{2.5,-2.5}{1}    % Control
\placewithtokens{p4}{7.5,-2.5}{0}    % Output
\placewithtokens{p5}{2.5,-5.5}{0}    % Complete

% Define transitions with new coordinates
\node[transition] (t1) at (2.5,0) {};   % Start (between P1 and P2, above P3)
\node[transition] (t2) at (5,-2.5) {};  % Execute (between P2 and P4, aligned with P3)
\node[transition] (t3) at (1,-4) {};    % Cancel (below and to the left of P3, leading to P5)
\node[transition] (t4) at (4,-4) {};    % Finish (below and to the right of P3, or below T2, leading to P5)

% Define arcs (connections remain logically the same)
\draw[pre] (p1) -- (t1);
\draw[post] (t1) -- (p2);
\draw[pre] (p3) -- (t1);    % P3 is an input to T1
\draw[post] (t1) -- (p3);   % T1 returns a token to P3 (maintains control token)

\draw[pre] (p2) -- (t2);
\draw[pre] (p3) -- (t2);    % P3 is also an input to T2
\draw[post] (t2) -- (p4);

\draw[pre] (p3) -- (t3);
\draw[post] (t3) -- (p5);

\draw[pre] (p4) -- (t4);
\draw[post] (t4) -- (p5);

% Add labels (node anchors ensure they are placed relative to the node)
\node[above] at (p1.north) {Resource};
\node[above] at (p2.north) {Process};
\node[left] at (p3.west) {Control}; % Adjusted to left to avoid overlap with T1-P3 arc
\node[above] at (p4.north) {Output}; % Adjusted to above for consistency
\node[below] at (p5.south) {Complete};

\node[above] at (t1.north) {Start};
\node[above] at (t2.north) {Execute}; % Adjusted to above
\node[left] at (t3.west) {Cancel};
\node[right] at (t4.east) {Finish};

% Add a title
\node[above=1cm of p1.north -| t1.north] {Petri Net Model of a Simple Workflow Process}; % Position title relative to top elements

\end{tikzpicture}