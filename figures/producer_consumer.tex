% Petri Net Model of the Producer-Consumer Problem (Refined)
\begin{tikzpicture}[node distance=2.2cm and 2.2cm, >=stealth, bend angle=30, auto, on grid]
    % Places
    \placewithtokens{producer}{0,0}{1}
    \placewithtokens{buffer}{2.8,0}{0}
    \placewithtokens{consumer}{5.6,0}{1}
    \placewithtokens{bufferCapacity}{2.8,-2.2}{3}

    % Transitions
    \node[transition, rounded corners=2pt] (produce) [right=of producer] {};
    \node[transition, rounded corners=2pt] (consume) [right=of buffer] {};

    % Arcs
    \draw[pre] (producer) -- (produce);
    \draw[post] (produce) -- (producer);
    \draw[post] (produce) -- (buffer);
    \draw[pre] (buffer) -- (consume);
    \draw[pre] (consumer) -- (consume);
    \draw[post] (consume) -- (consumer);
    \draw[pre] (bufferCapacity) -- (produce);
    \draw[post] (consume) -- (bufferCapacity);

    % Labels
    \node[above] at (producer.north) {Producer};
    \node[above] at (buffer.north) {Buffer};
    \node[above] at (consumer.north) {Consumer};
    \node[below] at (bufferCapacity.south) {Buffer Capacity};
    \node[above] at (produce.north) {Produce};
    \node[above] at (consume.north) {Consume};

    % Title
    \node[above=1cm] at (current bounding box.north) {Petri Net Model of the Producer-Consumer Problem};
\end{tikzpicture}