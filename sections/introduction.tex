\sectionheader{Introduction}

We propose a novel synthesis of Petri Nets and tensor algebra, arguing that Petri Nets can leverage the mathematical formalism of tensors to represent topologically related operands and their interactions. Drawing on David Spivak’s theory of polynomial functors, we introduce the concept of a Tensorized Petri Net, where each element can act simultaneously as an operand and an operator. We illustrate this framework using digit-wise arithmetic, showing how it enables compositional, interpretable, and parallelizable symbolic computation.

Petri Nets are a foundational tool for modeling distributed and concurrent systems, providing a graphical and mathematical language for representing state, transitions, and resource flow. Traditionally, Petri Nets are used to model discrete events and token flows, but recent advances in neural-symbolic computation and category theory suggest new ways to enrich their expressive power.

In this work, we argue that Petri Nets can be “tensorized”—that is, their places, transitions, and token flows can be represented and manipulated using tensor algebra. This enables the explicit modeling of topological relationships among operands, such as those found in digit-wise arithmetic or spatially structured data. Furthermore, by leveraging David Spivak’s ideas on polynomial functors, we can treat every element in the net as both an operand and an operator, capturing higher-order compositionality and self-similarity.

The main components of Petri Nets include:
\begin{itemize}
    \item Places (represented as circles)
    \item Transitions (represented as rectangles)
    \item Arcs (directed edges connecting places to transitions or transitions to places)
    \item Tokens (represented as dots within places)
\end{itemize}

In this paper, we explore the application of Petri Nets to model [specific system or process], with a focus on [specific aspect or property]. % In the section where you list contributions:

Our contributions include:
\begin{itemize}[leftmargin=*,align=left]
    \item \textbf{Contribution 1:} A novel approach to modeling concurrent processes using Petri Nets.
    \item \textbf{Contribution 2:} Analysis of deadlock properties in the context of resource allocation.
    \item \textbf{Contribution 3:} Implementation and evaluation of the proposed model using simulation.
\end{itemize}

// Remove the duplicate contributions list
% The main contributions of this work include:
% \begin{itemize}[leftmargin=*,align=left,widest=Contribution 3]
%   \item \textbf{Contribution 1:} Description of your first contribution...
%   \item \textbf{Contribution 2:} Description of your second contribution...
%   \item \textbf{Contribution 3:} Description of your third contribution...
% \end{itemize}

The remainder of this paper is organized as follows: Section \ref{sec:background} provides background information and related work. Section \ref{sec:methodology} describes our methodology and the proposed Petri Net model. Section \ref{sec:results} presents the results and analysis. Finally, Section \ref{sec:conclusion} concludes the paper and discusses future work.