\sectionheader{Introduction}

Petri Nets are a powerful mathematical modeling language used for the description of distributed systems \cite{murata1989petri}. They provide a graphical notation for stepwise processes that include choice, iteration, and concurrent execution. As a modeling language, Petri Nets offer a way to visually represent the structure of a distributed system as a directed bipartite graph with annotations.

The main components of Petri Nets include:
\begin{itemize}
    \item Places (represented as circles)
    \item Transitions (represented as rectangles)
    \item Arcs (directed edges connecting places to transitions or transitions to places)
    \item Tokens (represented as dots within places)
\end{itemize}

In this paper, we explore the application of Petri Nets to model [specific system or process], with a focus on [specific aspect or property]. % In the section where you list contributions:

Our contributions include:
\begin{itemize}[leftmargin=*,align=left]
    \item \textbf{Contribution 1:} A novel approach to modeling concurrent processes using Petri Nets.
    \item \textbf{Contribution 2:} Analysis of deadlock properties in the context of resource allocation.
    \item \textbf{Contribution 3:} Implementation and evaluation of the proposed model using simulation.
\end{itemize}

// Remove the duplicate contributions list
% The main contributions of this work include:
% \begin{itemize}[leftmargin=*,align=left,widest=Contribution 3]
%   \item \textbf{Contribution 1:} Description of your first contribution...
%   \item \textbf{Contribution 2:} Description of your second contribution...
%   \item \textbf{Contribution 3:} Description of your third contribution...
% \end{itemize}

The remainder of this paper is organized as follows: Section \ref{sec:background} provides background information and related work. Section \ref{sec:methodology} describes our methodology and the proposed Petri Net model. Section \ref{sec:results} presents the results and analysis. Finally, Section \ref{sec:conclusion} concludes the paper and discusses future work.