\sectionheader{Background and Related Work}
\label{sec:background}

\subsection{Petri Net Fundamentals}

Formally, a Petri Net is a tuple $(P, T, F, M_0)$ where:
\begin{itemize}
    \item $P$ is a finite set of places
    \item $T$ is a finite set of transitions
    \item $F \subseteq (P \times T) \cup (T \times P)$ is a set of arcs
    \item $M_0: P \rightarrow \mathbb{N}$ is the initial marking
\end{itemize}

The dynamics of Petri Nets are governed by the firing of transitions. A transition $t$ is enabled if each input place $p$ has at least as many tokens as the weight of the arc from $p$ to $t$. When a transition fires, it consumes tokens from its input places and produces tokens in its output places according to the weights of the corresponding arcs.

\subsection{Types of Petri Nets}

Several extensions to the basic Petri Net model have been proposed to enhance its modeling power:

\begin{itemize}
    \item \textbf{Colored Petri Nets} \cite{jensen1987coloured}: Extend the basic model by allowing tokens to have data values (colors).
    \item \textbf{Timed Petri Nets}: Incorporate time into the model, allowing for the analysis of temporal properties.
    \item \textbf{Stochastic Petri Nets}: Introduce probabilistic elements to model random behavior.
    \item \textbf{Hierarchical Petri Nets}: Allow for the decomposition of complex models into simpler submodels.
\end{itemize}

\subsection{Applications of Petri Nets}

Petri Nets have been applied to various domains, including:

\begin{itemize}
    \item \textbf{Workflow Management} \cite{van2000workflow}: Modeling and analyzing business processes.
    \item \textbf{Manufacturing Systems}: Modeling production lines and resource allocation.
    \item \textbf{Communication Protocols}: Analyzing the behavior of network protocols.
    \item \textbf{Software Design}: Modeling concurrent and distributed software systems.
\end{itemize}

\subsection{Related Work}

[Include relevant related work here, citing appropriate references from your bibliography]