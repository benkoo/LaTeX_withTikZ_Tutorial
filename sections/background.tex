\section{Foundational Principles}\label{sec:background}

\subsection{The Digit-Wise Processing Paradigm}

At its core, GASing employs a digit-wise processing approach that breaks numbers into their constituent digits and processes them systematically. This approach aligns with both human cognitive abilities and computational architecture:

\begin{itemize}
    \item \textbf{Human Cognition}: Processing single-digit operations recruits well-established neural pathways and memory systems \cite{dehaene2011}.
    \item \textbf{Computational Architecture}: Digit-wise processing maps naturally to register-based operations in modern processors.
\end{itemize}

\subsection{Modular Operation Design}

GASing builds arithmetic operations as modular extensions of one another:

\begin{enumerate}
    \item \textbf{Addition} serves as the foundational operation
    \item \textbf{Multiplication} is constructed as specialized, repeated addition
    \item \textbf{Subtraction} is implemented as addition with complementary digits
    \item \textbf{Division} is approached through repeated subtraction with optimization
\end{enumerate}

This modularity creates a coherent framework where mastery of one operation directly facilitates understanding of others.

\subsection{Lookup Tables and Pattern Recognition}

Central to the GASing method is the use of precomputed lookup tables for basic operations, particularly in addition. These tables:

\begin{itemize}
    \item Reduce cognitive load by eliminating the need for real-time calculation of base operations
    \item Enable pattern recognition across multiple calculations
    \item Function similarly to CPU cache mechanisms in computing systems
\end{itemize}
