The GASing method is built upon several coherent principles designed to demonstrate the transformative power of \textbf{composition and decomposition through digit-wise manipulation of representation systems}. At its essence, GASing provides both a rigorous formalism and an intuitive mental model of resource-awareness, showing how complex arithmetic operations can be systematically constructed from and deconstructed into elemental addition steps. By grounding all arithmetic in compositional patterns of addition, GASing makes resource consumption explicitly visible and quantifiable at every step of the reasoning process. Crucially, this approach establishes a \textbf{resource-aware axiomatic foundation} that derives its power not from statistical patterns or dataset-dependent heuristics, but from the fundamental mathematical properties of addition operations and their compositional nature at the representational level. The resulting framework enables practitioners to perceive, measure, and optimize computational resources with unprecedented clarity—whether those resources are CPU cycles, memory units, or human cognitive capacity. This axiomatic approach to composition ensures that computational correctness and resource efficiency are achieved through principled rules rather than empirical biases, making the system inherently trustworthy across all possible inputs and contexts.
\paragraph{The Digit-Wise Processing Paradigm}

At its core, GASing employs a digit-wise processing approach that treats numerical segments as \textbf{cell-like modules} within a larger computational organism. This approach doesn't merely break numbers into fixed constituent digits but rather establishes a \textbf{progressive function application} framework where each digit or digit-group functions as an independent computational unit. These units interact through well-defined interfaces (carry/borrow propagations) while maintaining their modular integrity, much like biological cells function within a larger organism.

The boundaries of these arithmetic operations are flexibly defined by utilizing the topological properties of Carry and Borrow propagations between adjacent numerical segments. This creates a \textbf{self-similar computational architecture} where the same fundamental operations apply at multiple scales, from single-digit operations to complex multi-segment calculations. The granularity of these segments can be dynamically determined to optimize arithmetic calculation efficiency, particularly leveraging the observation that operations become highly efficient when all unique combinations of numerical values within these segments can be assessed \emph{a priori}. 

A critical advantage of this approach is its exploitation of \textbf{combinatorial patterns in self-similar systems}. For any given segment size, there exists only a finite set of combinatorial possibilities among pairs of addition operands. These combinations form a \textbf{combinatorial space} that exhibits fractal-like properties—the same patterns of interaction between digits recur at different scales of magnitude. By pre-calculating and storing these patterns in optimized lookup tables, GASing achieves \textbf{computational compression}, dramatically reducing redundant calculations during runtime.

This pre-calculation strategy creates a \textbf{progressive application pipeline} where frequently encountered patterns are processed through highly optimized lookup operations, while less common combinations fall back to more general computational pathways. In contexts like Large Language Model (LLM) inference, where numerical patterns often follow predictable distributions, this approach enables \textbf{adaptive computation}—allocating more resources to common operations while gracefully handling edge cases. The system's ability to tailor its computational approach to both the statistical properties of the input and the available hardware resources (cache sizes, memory hierarchies, etc.) allows for acceleration that scales efficiently with problem complexity.

This refined digit-wise paradigm represents a \textbf{progressive function application} framework that minimizes operational complexity through \textbf{cellular decomposition}. Each digit or digit-group functions as a semi-autonomous computational unit, processing information according to a consistent set of rules while maintaining the flexibility to adapt to different contexts and scales. This approach transforms arithmetic from a monolithic operation into a \textbf{distributed computation} where each cell (digit) contributes to the overall result through local interactions with its immediate neighbors.

By focusing on these \textbf{digit-wise cellular modules}, GASing achieves a form of \textbf{computational compression} where complex operations emerge from the interaction of simpler, well-defined components. This approach aligns with both human cognitive abilities and modern computational architecture through several key principles:

- \textbf{Human Cognition}: By processing operations at the level of these flexibly defined numerical segments (which can be as small as single digits), GASing leverages established neural pathways for simpler operations (Dehaene, 2011). This modular, segment-based processing keeps the cognitive load for each step manageable, making the system intuitive and easier for human users to verify, even as the definition of a 'segment' adapts for efficiency.
- \textbf{Computational Architecture}: This segment-wise processing, where segments can be optimized for computational efficiency (e.g., to fit register sizes or leverage pre-computed lookup tables for segment-level operations), maps effectively to modern processor capabilities. The ability to dynamically adjust segment sizes and lookup table structures based on the specific computational environment enables optimal utilization of hardware resources, from register sizes and cache hierarchies to SIMD capabilities. Particularly noteworthy is how GASing naturally aligns with \textbf{\textit{SIMD Within A Register}} (\textbf{\textit{SWAR}}) techniques, which allow multiple small values to be processed simultaneously within a single processor register. By carefully selecting segment sizes to match register subword boundaries, GASing can exploit parallelism at the instruction level without requiring special vector instructions. This approach is highly dependent on the numeric values of the inputs—specific digit patterns can be arranged to avoid carry chains between segments, enabling true parallel processing of multiple additions within a single register. For example, with 64-bit registers, eight 8-bit segments can be processed simultaneously if the operands are properly arranged to prevent inter-segment carry propagation. Reducing operations to their simplest form at the segment level (e.g., segment addition and inter-segment carry/borrow) allows for granular understanding, control of computational resources, and potential for optimized hardware implementations that can adaptively select between scalar, SWAR, or full SIMD execution paths based on the specific digit patterns present in the inputs.

This fine-grained processing is key to building complex operations from the simplest possible base, ensuring that the entire system remains transparent and its resource demands predictable, while still offering substantial performance benefits through strategic pre-calculation and lookup table optimization.

Unlike approaches that depend on statistical regularities in training data, GASing's axiomatic foundation operates independently of any dataset biases. Its resource awareness stems from intrinsic mathematical properties rather than empirical patterns, providing a universal verification mechanism that remains valid across all possible numerical inputs. This makes the system particularly valuable in contexts requiring formal guarantees of correctness and predictable resource consumption. By embedding resource accounting directly into the axioms of addition operations, GASing creates a computational framework where efficiency and verifiability are inherent properties of the system itself, not artifacts of specific implementation choices or training procedures. This foundational commitment to resource-aware axiomatic rules ensures that GASing can serve as a trustworthy computational substrate even in novel domains or with inputs that fall outside the distribution of previously encountered patterns.
\paragraph{Modular Operation Design: A Minimal Vocabulary Anchored in Addition}

GASing builds all arithmetic operations as modular extensions of one another, with \textbf{addition, applied at the level of flexibly defined numerical segments, serving as the single, foundational operator}. This hierarchical construction, leveraging the segment-wise processing described earlier, is central to achieving a minimal operational vocabulary and directly impacts the assessment of resource consumption:

1.  \textbf{Segment-wise Addition} serves as the foundational, irreducible operation. All other arithmetic operations are defined in terms of sequences or transformations of this fundamental segment-wise addition, including the management of carry and borrow propagations between segments.
2.  \textbf{Multiplication} is constructed as specialized, repeated segment-wise addition. The process involves systematic application of segment-level additions and accumulation of partial results, explicitly defining multiplication's resource cost in terms of the underlying additive operations on segments.
3.  \textbf{Subtraction} is implemented as segment-wise addition using complementary segment values (e.g., employing ten's complement for decimal segments or two's complement for binary segments). This reframes subtraction entirely within the additive framework at the segment level, maintaining the minimal operational vocabulary.
4.  \textbf{Division} is approached through repeated segment-wise subtraction (which, as noted, is itself addition-based) with optimizations that can leverage pattern recognition across segments. Its complexity and resource use are, therefore, also traceable back to the fundamental segment-wise addition operations.

This modularity, centered on segment-wise addition, creates a coherent and parsimonious framework. Mastery of segment-wise addition directly facilitates the understanding and implementation of all other operations. More importantly, it means that the entire arithmetic system can be analyzed, and its resource consumption (both cognitive and computational) can be estimated based on the number and type of segment-wise addition-equivalent steps involved. This contrasts sharply with systems where each operator might be a black box with unique, opaque resource demands, and it aligns with the goal of transparently assessing computational effort by reducing all operations to a common, addition-based denominator at a flexible granularity.
\paragraph{Lookup Tables and Pattern Recognition: Optimizing the Core Operator}

Central to the GASing method is the use of precomputed lookup tables for basic operations, \textbf{particularly for single-digit addition and its immediate consequences (like carry generation)}. These tables are not an expansion of the operational vocabulary but rather an optimization strategy for the core addition operator:

-   \textbf{Reduce Cognitive and Computational Load}: By pre-calculating and storing the results of all possible single-digit additions (e.g., 0+0 through 9+9), the need for real-time calculation of these base operations is eliminated. This directly speeds up the execution of the foundational operator.
-   \textbf{Enable Pattern Recognition}: Consistent use of lookup tables for the core additive step allows for the easier identification of recurring patterns across multiple calculations. This can lead to higher-level optimizations and a better understanding of the computational structure of a problem, all while still operating within an addition-centric framework.
-   \textbf{Analogous to Caching}: These tables function similarly to CPU cache mechanisms or memoization in computing systems, storing frequently accessed results to avoid redundant computation. This makes the core addition process highly efficient.

By optimizing the execution of the single core operator (addition) through lookup tables, GASing ensures that the minimal vocabulary does not come at the cost of prohibitive inefficiency for elementary steps. This focus on optimizing the fundamental building block is crucial for the scalability and practicality of the approach, ensuring that even complex reasoning built from these simple steps remains manageable in terms of resource consumption.

This approach has direct implications for modern deep learning architectures, particularly Large Language Models (LLMs). Just as techniques like Word2Vec and Model2Vec pre-compute vector representations of tokens to accelerate natural language processing tasks, GASing pre-computes arithmetic operations and save the results in lookup tables to accelerate numerical reasoning. The Model2Vec approach, which can make sentence transformers up to 500x faster and 15x smaller by distilling uncontextualized token representations, demonstrates how strategic pre-computation can dramatically improve inferencing efficiency. Similarly, GASing's pattern-based optimizations (achieving up to 10x speedups for certain numerical patterns) reflect the same fundamental insight: by diving to an even lower level of digit-wise arithmetic optimization and adaptively leveraging the caching mechanisms of the computing hardware, significant performance improvements can be attained without any loss of arithmetic precision. This granular approach to numerical calculations enables GASing to exploit regularities in computational structures that would otherwise remain untapped in conventional implementations.

This connection extends to advanced transformer architectures like BERT and sequential recommendation systems like SASRec, which employ GATING mechanisms (the term for transformer attention operations) to selectively focus on relevant tokens or entities in a sequence. The sparse attention patterns that emerge in these models are conceptually akin to the selective application of computational resources that GASing employs. Just as transformers apply attention weights to determine which tokens are most relevant for a given context, GASing identifies which digit patterns warrant specialized handling. This parallel is more than superficial—both approaches fundamentally rely on identifying sparse patterns of relevance within a larger computational space. In this light, GASing can be viewed as an optimization process that works at the edge computing operation level, providing the fundamental arithmetic building blocks upon which higher-level sparse gating operations ultimately depend.

Performing arithmetic more efficiently has profound consequences for LLM inferencing operations, which rely heavily on matrix multiplications and other arithmetic-intensive computations. The patterns in which specific numerical values or operations occur during inference can be optimized using the same principles that GASing employs. This is especially relevant as models scale, where resource constraints become increasingly stringent. The choice of which reasoning approaches to apply—whether in human education or AI system design—is often driven by resource-bound concerns that can be optimized through pattern recognition of previously learned numerical value combinations. By formalizing this process through addition-centric operations with lookup-based acceleration, GASing provides both a theoretical framework and practical methodology for reasoning about and improving computational efficiency across domains.
