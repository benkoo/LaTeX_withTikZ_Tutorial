Division, in its most fundamental GASing form, is conceptualized as repeated subtraction. Given that subtraction itself is an additive operation (using complements), division becomes a higher-order construct built upon layers of addition.
\begin{itemize}
\item \textbf{Process:} To compute `a / b`, GASing repeatedly subtracts `b` (the divisor) from `a` (the dividend) using the `GASing_Subtraction` method. The number of successful subtractions before `a` becomes less than `b` (or zero) constitutes the quotient. The final value of `a` after these subtractions is the remainder.
\item \textbf{Optimization:} While simple repeated subtraction can be inefficient, GASing allows for optimizations. These can include subtracting multiples of `b` (e.g., `10\textit{b`, `100}b`), similar to long division, or leveraging pattern recognition to estimate parts of the quotient more quickly. However, even these optimized steps are ultimately resolved through sequences of the core `GASing_Subtraction` (and therefore `GASing_Addition`) operations.

\end{itemize}
\begin{verbatim}
def GASing_Division(dividend, divisor, base=10):
\begin{verbatim}
"""
Perform division using repeated GASing_Subtraction.
dividend, divisor: Strings representing non-negative integers.
base: Integer base for the operation (default: 10).
Returns (quotient: str, remainder: str)
"""
if divisor == pad_with_zeros("0", len(divisor)):
return "Error: Division by zero"
quotient = "0"
current_dividend = dividend
one = pad_with_zeros("1", len(quotient))
\end{verbatim}

\begin{verbatim}
\section{Comparison helper assumed: is_greater_or_equal(num1, num2)}
while is_greater_or_equal(current_dividend, divisor):
subtraction_result, is_neg = GASing_Subtraction(current_dividend, divisor, base)
if is_neg:
break
current_dividend = subtraction_result
quotient = GASing_Addition(quotient, one)
\section{Adjust padding for growing quotient}
if len(quotient) > len(one):
one = pad_with_zeros("1", len(quotient))
elif len(one) > len(quotient):
quotient = pad_with_zeros(quotient, len(one))
\end{verbatim}

\begin{verbatim}
remainder = current_dividend
return quotient, remainder
\end{verbatim}

\end{verbatim}

\textit{Helpers `pad_with_zeros`, `GASing_Addition`, and `is_greater_or_equal` are assumed to be defined and base-aware.}

---

By defining subtraction and division in terms of addition, GASing ensures that the entire arithmetic framework remains anchored to a single, fundamental operation. This not only simplifies the conceptual model but also provides a consistent basis for analyzing computational resource consumption, as all operations can be broken down into equivalent additive steps.
