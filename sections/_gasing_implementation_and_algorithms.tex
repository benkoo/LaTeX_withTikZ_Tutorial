\subsubsection{3.1 Addition Algorithm}

The GASing addition algorithm processes numbers from left to right (most significant digit to least significant), a departure from the traditional right-to-left approach. This design choice is deliberate and offers several advantages rooted in computational efficiency and cognitive alignment, particularly when considering the broader goals of the GASing framework to minimize operational vocabulary and optimize resource consumption.

A key principle underpinning this algorithm is its explicit leverage of \textbf{n-ary arithmetic}. The algorithm is designed to be agnostic to the base of the numerical segments being processed. Whether the system operates in binary, tertiary, decimal, hexadecimal, or any other base (n-ary), the core logic of left-to-right processing with carry propagation remains consistent. This flexibility allows the system to adapt the granularity of its operations (i.e., the 'digit' size or segment length) to best fit resource consumption optimization schemes. For instance, the segment size can be chosen to align with the cache line size of a processor or the optimal block size for memory access, thereby minimizing latency and maximizing throughput for pre-calculated and stored intermediate results.

This adaptability is crucial for applying GASing principles to complex reasoning activities, potentially even those embedded within advanced AI architectures like the \textbf{Sparse Autoencoders (SAEs)} described in the <mcfile name="ScalingMonoSemanticity.md" path="/Users/bkoo/Documents/Development/AIProjects/GASing\_PKM/docs/references/ScalingMonoSemanticity.md"></mcfile> paper. SAEs aim to decompose complex model activations into a sparse set of interpretable, monosemantic features. In essence, an SAE learns a large dictionary of these features, where only a small subset is active for any given input. This learned dictionary of features in an SAE can be seen as analogous to a highly optimized, distributed lookup table within the GASing framework. 

The GASing addition algorithm, by being designed for flexible n-ary arithmetic and optimized segment processing, aligns well with such architectures. If the 'features' learned by an SAE can be mapped to or interact with the numerical segments processed by GASing, then the pre-calculated operations and lookup tables inherent in GASing could significantly enhance the efficiency and interpretability of these SAEs. The left-to-right processing allows for incremental computation and potential early termination if an approximate result suffices, which can be beneficial in resource-constrained environments or when dealing with the vast feature spaces of SAEs. Furthermore, by designing arithmetic operations that can be efficiently cached and retrieved, GASing can support the rapid activation and combination of these 'semantic features' in an SAE, effectively making the SAE a powerful, dynamic dictionary that GASing can interact with for reasoning tasks.

The pseudo-code remains as a fundamental illustration:

\begin{verbatim}
function GASing_Addition(a, b):
    result = ""
    carry = 0
    
    # Pad the shorter number with leading zeros
    a = pad_with_zeros(a, len(b))
    b = pad_with_zeros(b, len(a))
    
    # Process from left to right
    for i in range(0, len(a)):
        # Add digits and carry
        digit_sum = int(a[i]) + int(b[i]) + carry
        
        # Determine new digit and carry
        if digit_sum > 9:
            carry = 1
            digit = digit_sum - 10
        else:
            carry = 0
            digit = digit_sum
            
        result += str(digit)
    
    # Add final carry if necessary
    if carry > 0:
        result += str(carry)
        
    return result

\end{verbatim}

This left-to-right, n-ary adaptable processing allows for:
-   \textbf{Flexible Resource Optimization:} Tailoring segment size (n-ary base) to hardware (cache, memory) or task demands.
-   \textbf{Alignment with Human Cognition:} Processing information sequentially, similar to reading.
-   \textbf{Potential for Parallelization:} Independent processing of segments once carries are managed.
-   \textbf{Integration with Learned Representations:} Provides a computational backend for systems like SAEs, where pre-calculated arithmetic on features (analogous to dictionary lookups) can speed up reasoning.
-   \textbf{Early Termination for Approximations:} Useful in iterative reasoning processes or when full precision is not immediately required.

By structuring the addition algorithm this way, GASing aims to provide a foundational arithmetic layer that is not only efficient in isolation but also highly compatible with modern AI architectures that rely on learned dictionaries and feature-based representations, such as Sparse Autoencoders.
\subsubsection{3.2 Multiplication Algorithm}

The GASing multiplication algorithm fundamentally extends the principles of the GASing addition operator, reframing multiplication as a systematic process of repeated, structured addition. It conceptualizes the multiplication of two numbers as the summation of partial products arranged in a grid-like structure. This approach not only maintains the core philosophy of minimizing operational vocabulary by grounding operations in addition but also enhances clarity and traceability.

Each cell in the conceptual grid represents the product of two individual digits (or segments, in n-ary arithmetic), which can be pre-calculated or retrieved from lookup tables, similar to single-digit additions. The core of the multiplication process then becomes the systematic summation of these grid values, column by column (or diagonal by diagonal, depending on the specific grid layout), applying the GASing addition algorithm (including its left-to-right carry propagation) to these intermediate sums. This effectively transforms multiplication into a series of additions, organized spatially by the grid.

The pseudo-code below illustrates this grid-based summation concept:



\begin{verbatim}
function GASing_Multiplication(a, b):
    # Initialize a grid to store partial products (results of single-digit multiplications)
    # The dimensions of the grid depend on the number of digits/segments in a and b.
    # For example, if a has M segments and b has N segments, grid is N x M.
        
    grid = initialize_partial_product_grid(a, b) # Each grid[i][j] = segment_b[i] * segment_a[j]
        
    # The core of multiplication is now to sum the values in this grid in a structured way.
    # This can be visualized as summing diagonals or columns, applying GASing addition.
    # For simplicity, imagine a function that collects these partial products and sums them
    # using the previously defined GASing_Addition logic, managing carries appropriately.

    final_product = "0"
    # Iterate through the grid, treating each row (or shifted row) as a number to be added.
    # This is a conceptual representation; actual implementation involves careful alignment and summation.
    for i in range(len(b)):
        partial_sum_for_row_i = "0"
        for j in range(len(a)):
            # Conceptually, each grid[i][j] contributes to a sum that is then added.
            # A more direct approach involves summing diagonals or columns with carries.
            # This step is a placeholder for the detailed grid summation logic.
            # For instance, grid[i][j] is like (digit_b[i] * digit_a[j]) * 10^(position_factor)
            # These terms are then summed up.
            pass # Detailed grid summation logic would be here.

    # A more accurate representation of grid summation:
    # 1. Calculate all single-segment products: product_ij = segment_a[j] * segment_b[i]
    # 2. Arrange these products in a grid, aligning them according to their place value.
    # 3. Sum the columns of this grid using GASing_Addition, propagating carries.

    # Example (conceptual): Summing diagonals of the grid of partial products
    # result = sum_grid_diagonals_with_gasing_addition(grid)

    # Simplified placeholder for the complex summation logic:
    # Assume 'grid_to_result' performs the systematic addition of partial products
    # according to GASing principles (left-to-right, carry management).
    
    result = perform_structured_addition_on_grid(grid, GASing_Addition_function_pointer)

    return result

# Helper function to demonstrate the addition algorithm
def perform_structured_addition_on_grid(grid, addition_algorithm):
    all_intermediate_sums = []
    for i in range(len(grid)):
        row_value_str = ""
        for val in grid[i]: # Assuming grid[i] is a list of partial products
            # Each row needs to be shifted appropriately before addition
            shifted_row_value = row_value_str + "0" * i # Conceptual shift
            all_intermediate_sums.append(shifted_row_value)

        current_total = "0"
    for num_str in all_intermediate_sums:
        current_total = addition_algorithm(current_total, num_str)
    return current_total

\end{verbatim}


The grid-based approach, when viewed as a structured application of the GASing addition algorithm, facilitates:

-   \textbf{Clear Visualization}: The multiplication process is broken down into a visible grid of elementary products (which are themselves results of lookup or minimal computation) and subsequent additions.
-   \textbf{Systematic Carry Handling}: Carries generated during the summation of grid elements are managed by the underlying GASing addition logic, ensuring consistency.
-   \textbf{Reinforcement of Additive Core}: Emphasizes that multiplication is not a fundamentally new operation but an organized, scaled-up application of addition.
-   \textbf{Identification of Patterns}: The structured grid can reveal patterns in partial products, which can be leveraged for optimization, especially when combined with n-ary segment processing and lookup tables for segment products.

By treating multiplication as an extension of addition via a grid, GASing maintains its commitment to a minimal operational vocabulary and enhances the interpretability of more complex arithmetic by tracing it back to foundational additive steps.
\subsubsection{3.3 Subtraction and Division: Extending Addition Further}

Consistent with GASing's core tenet of a minimal operational vocabulary, both subtraction and division are conceptualized and implemented as extensions of the foundational GASing addition algorithm.

\#### 3.3.1 Subtraction as Complemented Addition

Subtraction in GASing is performed by adding the complement of the subtrahend. For a given base (e.g., decimal or binary), the n's complement (e.g., ten's complement or two's complement) of the subtrahend is calculated and then added to the minuend using the \\texttt{GASing\_Addition} algorithm. This reframes subtraction entirely as an additive process, reinforcing the minimal operator set.

- \textbf{N's Complement:} The n's complement of a number \\texttt{b} with \\texttt{k} digits in base \\texttt{n} is \\texttt{(n^k - b)}. A common way to compute this is by finding the (n-1)'s complement (subtracting each digit from \\texttt{n-1}) and then adding 1 to the result.
- \textbf{Process:} To compute \\texttt{a - b}, GASing calculates \\texttt{a + (n's complement of b)}. If an overflow carry occurs from the most significant digit, it is typically discarded (in fixed-width representations), and the remaining result is the positive difference. If no overflow occurs, the result is negative, and its true magnitude is the n's complement of the sum, often flagged appropriately.

\begin{verbatim}
def GASing_Subtraction(a, b, base=10):
    """
    Perform subtraction using GASing's complemented addition approach.
    a, b: Strings representing non-negative integers.
    base: Integer base for the operation (default: 10).
    Returns (result: str, is_negative: bool)
    """
    num_digits = max(len(a), len(b))
    a_padded = pad_with_zeros(a, num_digits)
    b_padded = pad_with_zeros(b, num_digits)

    # (n-1)'s complement
    b_complement_n_minus_1 = ''.join(str((base - 1) - int(d)) for d in b_padded)
    # n's complement: add 1
    one_str = pad_with_zeros("1", num_digits)
    b_complement_n = GASing_Addition(b_complement_n_minus_1, one_str)

    # Ensure correct length after addition
    if len(b_complement_n) > num_digits and b_complement_n.startswith('1'):
        b_complement_n = b_complement_n[1:]
    else:
        b_complement_n = pad_with_zeros(b_complement_n, num_digits)

    # Add to minuend
    sum_with_complement = GASing_Addition(a_padded, b_complement_n)

    # Interpret result
    if len(sum_with_complement) > num_digits and sum_with_complement.startswith('1'):
        result = sum_with_complement[1:]
        is_negative = False
    else:
        if sum_with_complement == pad_with_zeros("0", num_digits):
            result = sum_with_complement
            is_negative = False
        else:
            # Take n's complement for magnitude
            temp_complement = ''.join(str((base - 1) - int(d)) for d in pad_with_zeros(sum_with_complement, num_digits))
            result = GASing_Addition(temp_complement, one_str)
            if len(result) > num_digits and result.startswith('1'):
                result = result[1:]
            else:
                result = pad_with_zeros(result, num_digits)
            is_negative = True
    return result, is_negative

\end{verbatim}

\emph{Helpers \\texttt{pad\_with\_zeros} and \\texttt{GASing\_Addition} are assumed to be defined elsewhere and base-aware.}

---

\#### 3.3.2 Division as Repeated Subtraction (Repeated Complemented Addition)

Division, in its most fundamental GASing form, is conceptualized as repeated subtraction. Given that subtraction itself is an additive operation (using complements), division becomes a higher-order construct built upon layers of addition.

- \textbf{Process:} To compute \\texttt{a / b}, GASing repeatedly subtracts \\texttt{b} (the divisor) from \\texttt{a} (the dividend) using the \\texttt{GASing\_Subtraction} method. The number of successful subtractions before \\texttt{a} becomes less than \\texttt{b} (or zero) constitutes the quotient. The final value of \\texttt{a} after these subtractions is the remainder.
- \textbf{Optimization:} While simple repeated subtraction can be inefficient, GASing allows for optimizations. These can include subtracting multiples of \\texttt{b} (e.g., \\texttt{10\emph{b}, \\texttt{100}b}), similar to long division, or leveraging pattern recognition to estimate parts of the quotient more quickly. However, even these optimized steps are ultimately resolved through sequences of the core \\texttt{GASing\_Subtraction} (and therefore \\texttt{GASing\_Addition}) operations.

\begin{verbatim}
def GASing_Division(dividend, divisor, base=10):
    """
    Perform division using repeated GASing_Subtraction.
    dividend, divisor: Strings representing non-negative integers.
    base: Integer base for the operation (default: 10).
    Returns (quotient: str, remainder: str)
    """
    if divisor == pad_with_zeros("0", len(divisor)):
        return "Error: Division by zero"
    quotient = "0"
    current_dividend = dividend
    one = pad_with_zeros("1", len(quotient))

    # Comparison helper assumed: is_greater_or_equal(num1, num2)
    while is_greater_or_equal(current_dividend, divisor):
        subtraction_result, is_neg = GASing_Subtraction(current_dividend, divisor, base)
        if is_neg:
            break
        current_dividend = subtraction_result
        quotient = GASing_Addition(quotient, one)
        # Adjust padding for growing quotient
        if len(quotient) > len(one):
            one = pad_with_zeros("1", len(quotient))
        elif len(one) > len(quotient):
            quotient = pad_with_zeros(quotient, len(one))

    remainder = current_dividend
    return quotient, remainder

\end{verbatim}

\emph{Helpers \\texttt{pad\_with\_zeros}, \\texttt{GASing\_Addition}, and \\texttt{is\_greater\_or\_equal} are assumed to be defined and base-aware.}

---

By defining subtraction and division in terms of addition, GASing ensures that the entire arithmetic framework remains anchored to a single, fundamental operation. This not only simplifies the conceptual model but also provides a consistent basis for analyzing computational resource consumption, as all operations can be broken down into equivalent additive steps.
