GASing builds all arithmetic operations as modular extensions of one another, with \textbf{addition, applied at the level of flexibly defined numerical segments, serving as the single, foundational operator}. This hierarchical construction, leveraging the segment-wise processing described earlier, is central to achieving a minimal operational vocabulary and directly impacts the assessment of resource consumption:

1.  \textbf{Segment-wise Addition} serves as the foundational, irreducible operation. All other arithmetic operations are defined in terms of sequences or transformations of this fundamental segment-wise addition, including the management of carry and borrow propagations between segments.
2.  \textbf{Multiplication} is constructed as specialized, repeated segment-wise addition. The process involves systematic application of segment-level additions and accumulation of partial results, explicitly defining multiplication's resource cost in terms of the underlying additive operations on segments.
3.  \textbf{Subtraction} is implemented as segment-wise addition using complementary segment values (e.g., employing ten's complement for decimal segments or two's complement for binary segments). This reframes subtraction entirely within the additive framework at the segment level, maintaining the minimal operational vocabulary.
4.  \textbf{Division} is approached through repeated segment-wise subtraction (which, as noted, is itself addition-based) with optimizations that can leverage pattern recognition across segments. Its complexity and resource use are, therefore, also traceable back to the fundamental segment-wise addition operations.

This modularity, centered on segment-wise addition, creates a coherent and parsimonious framework. Mastery of segment-wise addition directly facilitates the understanding and implementation of all other operations. More importantly, it means that the entire arithmetic system can be analyzed, and its resource consumption (both cognitive and computational) can be estimated based on the number and type of segment-wise addition-equivalent steps involved. This contrasts sharply with systems where each operator might be a black box with unique, opaque resource demands, and it aligns with the goal of transparently assessing computational effort by reducing all operations to a common, addition-based denominator at a flexible granularity.
% Content will be added here
