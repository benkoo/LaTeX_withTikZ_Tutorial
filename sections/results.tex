\sectionheader{Results and Discussion}
\label{sec:results}

The Tensorized Petri Net formalism offers several advantages:

\begin{itemize}
    \item \textbf{Parallelism:} Tensor operations enable efficient, parallel updates of the Petri Net state, mirroring the inherent concurrency of the net.
    \item \textbf{Compositionality:} Polynomial functors provide a principled way to compose and decompose Petri Net modules, supporting scalable and reusable designs.
    \item \textbf{Interpretability:} The explicit representation of operands, operators, and their topological relationships makes the computation transparent and analyzable.
    \item \textbf{Expressiveness:} This framework generalizes naturally to other structured computations, such as cellular automata, graph algorithms, and symbolic reasoning tasks.
\end{itemize}

\subsection{Model Validation}

[Describe how you validated your Petri Net model]

\subsection{Performance Analysis}

[Present the results of your performance analysis]

\begin{table}[htbp]
\centering
\caption{Performance Metrics for Different Configurations}
\label{tab:performance}
\begin{tabular}{@{}lccc@{}}
\toprule
\textbf{Metric} & \textbf{Config 1} & \textbf{Config 2} & \textbf{Config 3} \\
\midrule
Throughput & [value] & [value] & [value] \\
Response Time & [value] & [value] & [value] \\
Resource Utilization & [value] & [value] & [value] \\
\bottomrule
\end{tabular}
\end{table}

\subsection{Case Study}

[Present a case study demonstrating the application of your Petri Net model]

\subsection{Discussion}

[Discuss the implications of your results and any limitations of your approach]

\begin{figure}[htbp]
\centering
\resizebox{0.9\columnwidth}{!}{% Petri Net Model of the Producer-Consumer Problem (Refined)
\begin{tikzpicture}[node distance=2.2cm and 2.2cm, >=stealth, bend angle=30, auto, on grid]
    % Places
    \placewithtokens{producer}{0,0}{1}
    \placewithtokens{buffer}{2.8,0}{0}
    \placewithtokens{consumer}{5.6,0}{1}
    \placewithtokens{bufferCapacity}{2.8,-2.2}{3}

    % Transitions
    \node[transition, rounded corners=2pt] (produce) [right=of producer] {};
    \node[transition, rounded corners=2pt] (consume) [right=of buffer] {};

    % Arcs
    \draw[pre] (producer) -- (produce);
    \draw[post] (produce) -- (producer);
    \draw[post] (produce) -- (buffer);
    \draw[pre] (buffer) -- (consume);
    \draw[pre] (consumer) -- (consume);
    \draw[post] (consume) -- (consumer);
    \draw[pre] (bufferCapacity) -- (produce);
    \draw[post] (consume) -- (bufferCapacity);

    % Labels
    \node[above] at (producer.north) {Producer};
    \node[above] at (buffer.north) {Buffer};
    \node[above] at (consumer.north) {Consumer};
    \node[below] at (bufferCapacity.south) {Buffer Capacity};
    \node[above] at (produce.north) {Produce};
    \node[above] at (consume.north) {Consume};

    % Title
    \node[above=1cm] at (current bounding box.north) {Petri Net Model of the Producer-Consumer Problem};
\end{tikzpicture}}
\caption{Producer-Consumer Workflow Analysis}
\label{fig:producer_consumer_results}
\end{figure}