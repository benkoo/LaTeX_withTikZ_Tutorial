\sectionheader{Results and Discussion}
\label{sec:results}

The Tensorized Petri Net formalism offers several advantages:

\begin{itemize}
    \item \textbf{Parallelism:} Tensor operations enable efficient, parallel updates of the Petri Net state, mirroring the inherent concurrency of the net.
    \item \textbf{Compositionality:} Polynomial functors provide a principled way to compose and decompose Petri Net modules, supporting scalable and reusable designs.
    \item \textbf{Interpretability:} The explicit representation of operands, operators, and their topological relationships makes the computation transparent and analyzable.
    \item \textbf{Expressiveness:} This framework generalizes naturally to other structured computations, such as cellular automata, graph algorithms, and symbolic reasoning tasks.
\end{itemize}

\subsection{Model Validation}

[Describe how you validated your Petri Net model]

\subsection{Performance Analysis}

[Present the results of your performance analysis]

\begin{table}[htbp]
\centering
\caption{Performance Metrics for Different Configurations}
\label{tab:performance}
\begin{tabular}{@{}lccc@{}}
\toprule
\textbf{Metric} & \textbf{Config 1} & \textbf{Config 2} & \textbf{Config 3} \\
\midrule
Throughput & [value] & [value] & [value] \\
Response Time & [value] & [value] & [value] \\
Resource Utilization & [value] & [value] & [value] \\
\bottomrule
\end{tabular}
\end{table}

\subsection{Case Study}

[Present a case study demonstrating the application of your Petri Net model]

\subsection{Discussion}

[Discuss the implications of your results and any limitations of your approach]