The GASing arithmetic method introduces a foundational paradigm for numerical computation at the intersection of human cognition and artificial intelligence. In an era where AI systems increasingly shape and augment human reasoning, there is a critical need for computational frameworks that are not only transparent and interpretable but also capable of aligning with fundamental cognitive and organizational constraints. GASing addresses this challenge by establishing \textbf{addition as the universal meta-operator}—the cornerstone of a minimized operational vocabulary from which all arithmetic logic can be systematically constructed, analyzed, and verified. Through digit-wise processing and pattern recognition, GASing enables explicit quantification of computational effort and resource usage, bridging symbolic and neural approaches while supporting robust provenance tracking. This unified approach empowers both humans and machines to converge on a shared pattern of reasoning, fostering deeper alignment of intentions and facilitating the transfer and accumulation of knowledge across individuals, collectives, and artificial agents. Our findings demonstrate that GASing is not only a mathematically rigorous and pedagogically effective framework, but also a scalable substrate for interpretable, trustworthy, and resource-efficient AI—laying the groundwork for a unified theory of learning rooted in a common operator.
