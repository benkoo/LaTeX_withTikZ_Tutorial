The \textbf{\textit{GASing Arithmetic}} method introduces a foundational paradigm for numerical computation at the intersection of human cognition, cultural mathematics, and artificial intelligence. Drawing inspiration from Eglash's work in \textbf{\textit{Ethno Arithmetic}} and fractal patterns in indigenous knowledge systems, GASing demonstrates how culturally-grounded mathematical insights can inform modern computational frameworks. In an era where AI systems increasingly shape and augment human reasoning, there is a critical need for computational approaches that are not only transparent and interpretable but also capable of aligning with fundamental cognitive patterns across diverse cultural contexts. GASing addresses this challenge through the \textbf{progressive application of functions} and \textbf{digit-wise systems as cell-like modules}, establishing \textbf{addition as the universal meta-operator}—the cornerstone of a minimized operational vocabulary from which all arithmetic logic can be systematically constructed, analyzed, and verified. 

This approach leverages the inherent combinatorial patterns in self-similar systems to better compress unnecessary computational efforts. Through digit-wise processing and pattern recognition, GASing enables explicit quantification of computational effort and resource usage, bridging symbolic and neural approaches while supporting robust provenance tracking. The method's cell-like modular architecture, inspired by the fractal patterns found in traditional Indonesian design and other indigenous knowledge systems, allows for efficient information processing by breaking down complex operations into manageable, recombinable units. This unified approach has been successfully implemented in Indonesia's national education program, reaching millions of students and demonstrating how culturally-informed knowledge exchange protocols can be continuously improved at scale. Our findings demonstrate that GASing is not only a mathematically rigorous and pedagogically effective framework, but also a culturally-grounded substrate for interpretable, trustworthy, and resource-efficient AI—laying the groundwork for a unified theory of learning rooted in both universal mathematical principles and diverse cultural expressions of mathematical thinking.
