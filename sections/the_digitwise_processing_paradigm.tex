At its core, GASing employs a digit-wise processing approach that, instead of merely breaking numbers into fixed constituent digits, flexibly defines the boundaries of arithmetic operations by utilizing the topological properties of Carry and Borrow propagations between adjacent numerical segments (or 'digits'). The granularity of these segments can be dynamically determined to optimize arithmetic calculation efficiency, particularly leveraging the observation that operations become highly efficient when all unique combinations of numerical values within these segments can be assessed \textit{a priori}. This refined digit-wise paradigm is fundamental to minimizing the complexity of the operational vocabulary. Instead of treating numbers as holistic entities requiring a potentially tedious array of complex operations, GASing focuses on manipulating these individual segments using a very limited set of rules, primarily those governing the foundational principles of addition applied at the segment level. This approach aligns with both human cognitive abilities and computational architecture:
\begin{itemize}
\item \textbf{Human Cognition}: By processing operations at the level of these flexibly defined numerical segments (which can be as small as single digits), GASing leverages established neural pathways for simpler operations (Dehaene, 2011). This modular, segment-based processing keeps the cognitive load for each step manageable, making the system intuitive and easier for human users to verify, even as the definition of a 'segment' adapts for efficiency.
\item \textbf{Computational Architecture}: This segment-wise processing, where segments can be optimized for computational efficiency (e.g., to fit register sizes or leverage pre-computed lookup tables for segment-level operations), maps effectively to modern processor capabilities. Reducing operations to their simplest form at the segment level (e.g., segment addition and inter-segment carry/borrow) allows for granular understanding, control of computational resources, and potential for optimized hardware implementations.

\end{itemize}
This fine-grained processing is key to building complex operations from the simplest possible base, ensuring that the entire system remains transparent and its resource demands predictable.
% Content will be added here
