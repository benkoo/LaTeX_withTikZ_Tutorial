% Appendix
% This appendix contains supplementary material, proofs, extended algorithms, and additional data relevant to the GASing Arithmetic System.

\section{Appendix I: GASing Addition Algorithm}
\label{appendix:addition}
\begin{lstlisting}[language=Python,caption={GASing Addition Algorithm}]
function GASing_Addition(a, b):
    result = ""
    carry = 0
    
    # Pad the shorter number with leading zeros
    a = pad_with_zeros(a, len(b))
    b = pad_with_zeros(b, len(a))
    
    # Process from left to right
    for i in range(0, len(a)):
        # Add digits and carry
        digit_sum = int(a[i]) + int(b[i]) + carry
        
        # Determine new digit and carry
        if digit_sum > 9:
            carry = 1
            digit = digit_sum - 10
        else:
            carry = 0
            digit = digit_sum
        
        result += str(digit)
    
    # Add final carry if necessary
    if carry > 0:
        result += str(carry)
        
    return result
\end{lstlisting}

\section{Appendix II: GASing Multiplication Algorithm}
\label{appendix:multiplication}
\begin{lstlisting}[language=Python,caption={GASing Multiplication Algorithm (Conceptual)}]
function GASing_Multiplication(a, b):
    # Initialize a grid to store partial products (results of single-digit multiplications)
    # The dimensions of the grid depend on the number of digits/segments in a and b.
    # For example, if a has M segments and b has N segments, grid is N x M.
        
    grid = initialize_partial_product_grid(a, b) # Each grid[i][j] = segment_b[i] * segment_a[j]
        
    # The core of multiplication is now to sum the values in this grid in a structured way.
    # This can be visualized as summing diagonals or columns, applying GASing addition.
    # For simplicity, imagine a function that collects these partial products and sums them
    # using the previously defined GASing_Addition logic, managing carries appropriately.

    final_product = "0"
    # Iterate through the grid, treating each row (or shifted row) as a number to be added.
    # This is a conceptual representation; actual implementation involves careful alignment and summation.
    for i in range(len(b)):
        partial_sum_for_row_i = "0"
        for j in range(len(a)):
            # Conceptually, each grid[i][j] contributes to a sum that is then added.
            # A more direct approach involves summing diagonals or columns with carries.
            # This step is a placeholder for the detailed grid summation logic.
            # For instance, grid[i][j] is like (digit_b[i] * digit_a[j]) * 10^(position_factor)
            # These terms are then summed up.
            pass # Detailed grid summation logic would be here.

    # A more accurate representation of grid summation:
    # 1. Calculate all single-segment products: product_ij = segment_a[j] * segment_b[i]
    # 2. Arrange these products in a grid, aligning them according to their place value.
    # 3. Sum the columns of this grid using GASing_Addition, propagating carries.

    # Example (conceptual): Summing diagonals of the grid of partial products
    # result = sum_grid_diagonals_with_gasing_addition(grid)

    # Simplified placeholder for the complex summation logic:
    # Assume 'grid_to_result' performs the systematic addition of partial products
    # according to GASing principles (left-to-right, carry management).
    
    result = perform_structured_addition_on_grid(grid, GASing_Addition_function_pointer)

return result
\end{lstlisting}

\section{Appendix III: GASing Subtraction and Division Algorithms}
\label{appendix:subtraction}
\begin{lstlisting}[language=Python,caption={GASing Subtraction Algorithm}]
function GASing_Subtraction(a, b, base=10):
    # Ensure a and b are of the same length for complement calculation
    # This might involve padding the shorter number or defining a fixed width.
    # For simplicity, assume a and b are positive integers represented as strings.
    num_digits = max(len(a), len(b))

    a_padded = pad_with_zeros(a, num_digits)
    b_padded = pad_with_zeros(b, num_digits)

    # Calculate n's complement of b (e.g., ten's complement for base 10)
    # This involves (base^num_digits - b_padded)
    # A common method: ( (base-1)'s_complement of b_padded ) + 1
    b_complement_n_minus_1 = ""
    for digit_char in b_padded:
        b_complement_n_minus_1 += str((base - 1) - int(digit_char))

    # Add 1 to get n's complement (e.g., ten's complement)
    # This addition itself can use a simplified version of GASing_Addition or direct logic
    one_str = pad_with_zeros("1", num_digits)
    b_complement_n = GASing_Addition(b_complement_n_minus_1, one_str) # Assuming base compatibility
    # Handle potential overflow from complement addition if b_complement_n exceeds num_digits
    if len(b_complement_n) > num_digits and b_complement_n.startswith('1'): # Check for leading '1' from carry
        b_complement_n = b_complement_n[1:] # Keep only num_digits
    else:
        b_complement_n = pad_with_zeros(b_complement_n, num_digits)

    # Perform a + (n's complement of b)
    # The GASing_Addition function is used here.
    sum_with_complement = GASing_Addition(a_padded, b_complement_n) # Ensure base compatibility

    # Interpret the result
    if len(sum_with_complement) > num_digits and sum_with_complement.startswith('1'): # Overflow carry indicates positive result
        result = sum_with_complement[1:] # Discard overflow carry
        is_negative = False
    else:
        # No overflow carry indicates negative result or zero
        # The result is negative, and its magnitude is the n's complement of sum_with_complement
        # For simplicity, we'll just flag it as negative and take complement again for magnitude
        # (or handle based on specific complement arithmetic rules)
        if sum_with_complement == pad_with_zeros("0", num_digits):
            result = sum_with_complement
            is_negative = False
        else:
            # Recalculate n's complement of sum_with_complement to get magnitude
            temp_complement_n_minus_1 = ""
            sum_with_complement_padded = pad_with_zeros(sum_with_complement, num_digits)
            for digit_char in sum_with_complement_padded:
                temp_complement_n_minus_1 += str((base - 1) - int(digit_char))
            result = GASing_Addition(temp_complement_n_minus_1, one_str) # Magnitude
            if len(result) > num_digits and result.startswith('1'):
                result = result[1:]
            else:
                result = pad_with_zeros(result, num_digits)
            is_negative = True
            
    return result, is_negative
\end{lstlisting}

\begin{lstlisting}[language=Python,caption={GASing Division Algorithm}]
function GASing_Division(dividend, divisor, base=10):
    if divisor == pad_with_zeros("0", len(divisor)):
        return "Error: Division by zero"
    quotient = "0"
    current_dividend = dividend
    one = pad_with_zeros("1", len(quotient)) # For incrementing quotient

    # Repeatedly subtract divisor from current_dividend
    # We need a comparison function: is_greater_or_equal(num1, num2)
    while is_greater_or_equal(current_dividend, divisor):
        subtraction_result, is_neg = GASing_Subtraction(current_dividend, divisor, base)
        if is_neg: # Should not happen if is_greater_or_equal is correct
            break 
        current_dividend = subtraction_result
        quotient = GASing_Addition(quotient, one) # Increment quotient
        # Ensure 'one' is padded correctly if quotient grows
        if len(quotient) > len(one):
            one = pad_with_zeros("1", len(quotient))
        elif len(one) > len(quotient):
            quotient = pad_with_zeros(quotient, len(one))

    remainder = current_dividend
    return quotient, remainder
\end{lstlisting}

