The rise of large language models and neural computing has fundamentally transformed how humans reason in partnership with machines, yet this transformation brings new challenges in interpretability, verification, and cognitive alignment. Contemporary AI systems often operate through complex, opaque transformations, underscoring the need for foundational mechanisms that can bridge human cognition and computational processes. The GASing method responds to this need by reconceptualizing arithmetic operations through a digit-wise, pattern-recognition framework that serves as a universal computational paradigm—\textbf{anchored in the principle of minimizing its operational vocabulary by grounding all arithmetic in the fundamental operator of addition.}

By positioning addition as a meta-operator—capable of expressing the meaning of any symbolic token structure—GASing creates a language-agnostic verification framework that is both interpretable and measurable. This reduction to a core operator allows for direct assessment of computational and cognitive resource consumption, transparent verification of reasoning, and systematic arrangement of operations that parallel both transformer attention mechanisms and human chunking strategies. The result is a unified, cross-referential decision framework in which every logical step is both auditable and adaptable to the cognitive limits of users or the architectural constraints of machines.

The name "GASing" derives from Indonesian terms: Gampang (easy), Asyik (enjoyable), and menyenangkan (interesting), reflecting the method’s dual purpose: making arithmetic accessible for human understanding while providing a rigorous foundation for interpretable AI. Originally developed as a pedagogical tool, GASing now stands as a scalable, mathematically rigorous, and philosophically grounded approach—one that not only mirrors key aspects of modern AI architectures, but also fosters the convergence of human, machine, and organizational reasoning on a shared, minimal, and interpretable operational substrate. This commitment to a common operator is both a practical solution and a necessary condition for the emergence of a unified theory of learning and collaborative intelligence.
